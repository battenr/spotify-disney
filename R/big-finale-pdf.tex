% Options for packages loaded elsewhere
\PassOptionsToPackage{unicode}{hyperref}
\PassOptionsToPackage{hyphens}{url}
%
\documentclass[
]{article}
\usepackage{lmodern}
\usepackage{amssymb,amsmath}
\usepackage{ifxetex,ifluatex}
\ifnum 0\ifxetex 1\fi\ifluatex 1\fi=0 % if pdftex
  \usepackage[T1]{fontenc}
  \usepackage[utf8]{inputenc}
  \usepackage{textcomp} % provide euro and other symbols
\else % if luatex or xetex
  \usepackage{unicode-math}
  \defaultfontfeatures{Scale=MatchLowercase}
  \defaultfontfeatures[\rmfamily]{Ligatures=TeX,Scale=1}
\fi
% Use upquote if available, for straight quotes in verbatim environments
\IfFileExists{upquote.sty}{\usepackage{upquote}}{}
\IfFileExists{microtype.sty}{% use microtype if available
  \usepackage[]{microtype}
  \UseMicrotypeSet[protrusion]{basicmath} % disable protrusion for tt fonts
}{}
\makeatletter
\@ifundefined{KOMAClassName}{% if non-KOMA class
  \IfFileExists{parskip.sty}{%
    \usepackage{parskip}
  }{% else
    \setlength{\parindent}{0pt}
    \setlength{\parskip}{6pt plus 2pt minus 1pt}}
}{% if KOMA class
  \KOMAoptions{parskip=half}}
\makeatother
\usepackage{xcolor}
\IfFileExists{xurl.sty}{\usepackage{xurl}}{} % add URL line breaks if available
\IfFileExists{bookmark.sty}{\usepackage{bookmark}}{\usepackage{hyperref}}
\hypersetup{
  pdftitle={90's Disney Hits},
  pdfauthor={Ryan Batten},
  hidelinks,
  pdfcreator={LaTeX via pandoc}}
\urlstyle{same} % disable monospaced font for URLs
\usepackage[margin=1in]{geometry}
\usepackage{color}
\usepackage{fancyvrb}
\newcommand{\VerbBar}{|}
\newcommand{\VERB}{\Verb[commandchars=\\\{\}]}
\DefineVerbatimEnvironment{Highlighting}{Verbatim}{commandchars=\\\{\}}
% Add ',fontsize=\small' for more characters per line
\usepackage{framed}
\definecolor{shadecolor}{RGB}{248,248,248}
\newenvironment{Shaded}{\begin{snugshade}}{\end{snugshade}}
\newcommand{\AlertTok}[1]{\textcolor[rgb]{0.94,0.16,0.16}{#1}}
\newcommand{\AnnotationTok}[1]{\textcolor[rgb]{0.56,0.35,0.01}{\textbf{\textit{#1}}}}
\newcommand{\AttributeTok}[1]{\textcolor[rgb]{0.77,0.63,0.00}{#1}}
\newcommand{\BaseNTok}[1]{\textcolor[rgb]{0.00,0.00,0.81}{#1}}
\newcommand{\BuiltInTok}[1]{#1}
\newcommand{\CharTok}[1]{\textcolor[rgb]{0.31,0.60,0.02}{#1}}
\newcommand{\CommentTok}[1]{\textcolor[rgb]{0.56,0.35,0.01}{\textit{#1}}}
\newcommand{\CommentVarTok}[1]{\textcolor[rgb]{0.56,0.35,0.01}{\textbf{\textit{#1}}}}
\newcommand{\ConstantTok}[1]{\textcolor[rgb]{0.00,0.00,0.00}{#1}}
\newcommand{\ControlFlowTok}[1]{\textcolor[rgb]{0.13,0.29,0.53}{\textbf{#1}}}
\newcommand{\DataTypeTok}[1]{\textcolor[rgb]{0.13,0.29,0.53}{#1}}
\newcommand{\DecValTok}[1]{\textcolor[rgb]{0.00,0.00,0.81}{#1}}
\newcommand{\DocumentationTok}[1]{\textcolor[rgb]{0.56,0.35,0.01}{\textbf{\textit{#1}}}}
\newcommand{\ErrorTok}[1]{\textcolor[rgb]{0.64,0.00,0.00}{\textbf{#1}}}
\newcommand{\ExtensionTok}[1]{#1}
\newcommand{\FloatTok}[1]{\textcolor[rgb]{0.00,0.00,0.81}{#1}}
\newcommand{\FunctionTok}[1]{\textcolor[rgb]{0.00,0.00,0.00}{#1}}
\newcommand{\ImportTok}[1]{#1}
\newcommand{\InformationTok}[1]{\textcolor[rgb]{0.56,0.35,0.01}{\textbf{\textit{#1}}}}
\newcommand{\KeywordTok}[1]{\textcolor[rgb]{0.13,0.29,0.53}{\textbf{#1}}}
\newcommand{\NormalTok}[1]{#1}
\newcommand{\OperatorTok}[1]{\textcolor[rgb]{0.81,0.36,0.00}{\textbf{#1}}}
\newcommand{\OtherTok}[1]{\textcolor[rgb]{0.56,0.35,0.01}{#1}}
\newcommand{\PreprocessorTok}[1]{\textcolor[rgb]{0.56,0.35,0.01}{\textit{#1}}}
\newcommand{\RegionMarkerTok}[1]{#1}
\newcommand{\SpecialCharTok}[1]{\textcolor[rgb]{0.00,0.00,0.00}{#1}}
\newcommand{\SpecialStringTok}[1]{\textcolor[rgb]{0.31,0.60,0.02}{#1}}
\newcommand{\StringTok}[1]{\textcolor[rgb]{0.31,0.60,0.02}{#1}}
\newcommand{\VariableTok}[1]{\textcolor[rgb]{0.00,0.00,0.00}{#1}}
\newcommand{\VerbatimStringTok}[1]{\textcolor[rgb]{0.31,0.60,0.02}{#1}}
\newcommand{\WarningTok}[1]{\textcolor[rgb]{0.56,0.35,0.01}{\textbf{\textit{#1}}}}
\usepackage{graphicx,grffile}
\makeatletter
\def\maxwidth{\ifdim\Gin@nat@width>\linewidth\linewidth\else\Gin@nat@width\fi}
\def\maxheight{\ifdim\Gin@nat@height>\textheight\textheight\else\Gin@nat@height\fi}
\makeatother
% Scale images if necessary, so that they will not overflow the page
% margins by default, and it is still possible to overwrite the defaults
% using explicit options in \includegraphics[width, height, ...]{}
\setkeys{Gin}{width=\maxwidth,height=\maxheight,keepaspectratio}
% Set default figure placement to htbp
\makeatletter
\def\fps@figure{htbp}
\makeatother
\setlength{\emergencystretch}{3em} % prevent overfull lines
\providecommand{\tightlist}{%
  \setlength{\itemsep}{0pt}\setlength{\parskip}{0pt}}
\setcounter{secnumdepth}{-\maxdimen} % remove section numbering
\usepackage{multirow}
\usepackage{multicol}
\usepackage{colortbl}
\usepackage{hhline}
\usepackage{longtable}
\usepackage{array}
\usepackage{hyperref}

\title{90's Disney Hits}
\author{Ryan Batten}
\date{10/05/2021}

\begin{document}
\maketitle

\hypertarget{disney-hits-spotify-all-together}{%
\section{Disney Hits + Spotify: All
Together}\label{disney-hits-spotify-all-together}}

The aim of this project started as a NCAA March Madness style tournament
bracket. What began as a simple selection process, turned into a more
data-driven goal of ``this is good but how do we look at this with a
scientist's eye?''. Data are everywhere, but sometimes it requires a bit
more of ``playing around'' to understand what is really going on here.

\hypertarget{step-1-getting-setup-with-spotify}{%
\subsection{Step 1: Getting Setup with
Spotify}\label{step-1-getting-setup-with-spotify}}

If you have no clue how to do this, don't worry! I didn't either until I
started playing around with it. First step is to head over to the
amazing Thompson et al.~(2021) \emph{spotifyr} package to get setup with
a Spotify developer account.

\hypertarget{step-2-getting-familiar-with-the-data}{%
\subsection{Step 2: Getting Familiar with the
Data}\label{step-2-getting-familiar-with-the-data}}

So now you're in, what next? When you start exploring with the data it
can seem overwhelming. Lists on lists on lists\ldots.but not to fear
young Jedi! Where there is a will there is a way. This is the
exploratory data analysis phase, more of an art-form than a science
however can't recommend enough to check out ``R for Data Science'' by
Grolemund \& Wickham (2016). Like many before me, this is a great place
to start as a beginner.

We'll pick the four albums we want to examine (using Disney Hits we all
know and love!): The Lion King, Hercules, Tarzan and The Little Mermaid.

\hypertarget{step-3-alright-now-youre-familiar-whats-next}{%
\subsection{Step 3: Alright, now you're familiar what's
next?}\label{step-3-alright-now-youre-familiar-whats-next}}

This part requires a little more googling (a key part of everyone's
process: from data scientists to top level execs) to understand what is
exactly going on here. Once you have a Spotify developer account, feel
free to start poking around about what each component means: how the
data is stored, how to access it, etc. Spotify has some great, very
detailed guides on every aspect of what is in store.

\hypertarget{step-4-lets-get-the-album-id}{%
\subsection{Step 4: Lets Get the Album
ID}\label{step-4-lets-get-the-album-id}}

If you have the Spotify app already downloaded onto your desktop (like I
do), click on the album you'd like to use to pull the data from. You
will see three dots (\ldots): click share then ``Copy Album Link''. This
will give you the unique Spotify album ID. Again, using The Lion King
(1994) we end up with:
``\url{https://open.spotify.com/album/3YA5DdB3wSz4pdfEXoMyRd?si=CpCUiyLBT2ea5y7haYA9Cg}''.
The last string, that bizzare set of alphanumeric characters, after the
``album/'' are what is referred to as the Spotify album id.

\hypertarget{step-5-data-collection}{%
\subsection{Step 5: Data Collection}\label{step-5-data-collection}}

Time to get into it young grasshopper! We're going to use the
\emph{spotifyr::get\_album()} function to extract what we need here. It
will pull a \textbf{BUNCH} of meta-data but fear not. If we keep this
simple, we'll be just fine. Below is an example using The Lion King
(1994)

\begin{Shaded}
\begin{Highlighting}[]
\NormalTok{lion_king <-}\StringTok{ }\NormalTok{spotifyr}\OperatorTok{::}\KeywordTok{get_album}\NormalTok{(}\DataTypeTok{id =} \StringTok{"3YA5DdB3wSz4pdfEXoMyRd?si=9UUAysDuTS6ixCsXYMykWg"}\NormalTok{)}
\end{Highlighting}
\end{Shaded}

As you can see, each album is associated with a unique identifier
(typically a string that consists of a bunch of nonsensical letters!).
If they didn't make sense to you, don't worry they don't to me either!

Here is a list of the other three albums:

\begin{Shaded}
\begin{Highlighting}[]
\NormalTok{tarzan <-}\StringTok{ }\NormalTok{spotifyr}\OperatorTok{::}\KeywordTok{get_album}\NormalTok{(}\DataTypeTok{id =} \StringTok{'6fBzYwBKjuO4hmhcGuklJM?si=8G4ZAzUVRcm0FRF8aXphuw'}\NormalTok{) }\CommentTok{# Tarzan (1994)}
\NormalTok{little_mermaid <-}\StringTok{ }\NormalTok{spotifyr}\OperatorTok{::}\KeywordTok{get_album}\NormalTok{(}\DataTypeTok{id =} \StringTok{"4YTduhQWfS0pOzQC4o0HcG?si=LBpETiG2RSyZyIk8mloBQg"}\NormalTok{ ) }\CommentTok{# The Little Mermaid (1997)}
\NormalTok{hercules <-}\StringTok{ }\NormalTok{spotifyr}\OperatorTok{::}\KeywordTok{get_album}\NormalTok{(}\DataTypeTok{id =} \StringTok{"1wbY6VUchNsZLaDi22eD3J?si=00ajCSlHReuJXSz8XpARZg"}\NormalTok{) }\CommentTok{# Hercules (1997)}
\end{Highlighting}
\end{Shaded}

\hypertarget{step-6-popularity}{%
\subsection{Step 6: Popularity}\label{step-6-popularity}}

Now, we want to see how popular each of these songs are. There is a
variable labelled: popularity, which ranges from 0 to 100 with 100 being
the highest / most popular. Popularity is not static rather it is
dynamic, so it is liable to change over time. Anyways, at the time of
this writing: May 9, 2021 the results are (drummmmrollll pleaseee):

\begin{Shaded}
\begin{Highlighting}[]
\NormalTok{big_finale <-}\StringTok{ }\NormalTok{dplyr}\OperatorTok{::}\KeywordTok{tribble}\NormalTok{(}
  \OperatorTok{~}\NormalTok{album_name, }\OperatorTok{~}\NormalTok{popularity,}
  \StringTok{"Tarzan (1999)"}\NormalTok{, }\DecValTok{1}\NormalTok{,}
  \StringTok{"The Little Mermaid (1997)"}\NormalTok{, }\DecValTok{65}\NormalTok{,}
  \StringTok{"Hercules (1997)"}\NormalTok{, }\DecValTok{67}\NormalTok{,}
  \StringTok{"The Lion King (1994)"}\NormalTok{, }\DecValTok{76}
\NormalTok{)}

\NormalTok{big_finale <-}\StringTok{ }\NormalTok{big_finale }\OperatorTok\StringTok{ }
\StringTok{  }\NormalTok{dplyr}\OperatorTok{::}\KeywordTok{arrange}\NormalTok{(}
    \OperatorTok{-}\NormalTok{popularity }\CommentTok{# organizing table by popularity}
\NormalTok{  ) }

\NormalTok{big_finale_table <-}\StringTok{ }\NormalTok{flextable}\OperatorTok{::}\KeywordTok{flextable}\NormalTok{(big_finale) }\OperatorTok\StringTok{ }
\StringTok{  }\KeywordTok{set_header_labels}\NormalTok{(}\DataTypeTok{album_name =} \StringTok{"Album Name"}\NormalTok{, }
                    \DataTypeTok{popularity =} \StringTok{"Popularity"}\NormalTok{) }\OperatorTok\StringTok{ }
\StringTok{  }\NormalTok{flextable}\OperatorTok{::}\KeywordTok{theme_zebra}\NormalTok{()}
\end{Highlighting}
\end{Shaded}

\begin{Shaded}
\begin{Highlighting}[]
\NormalTok{big_finale_table}
\end{Highlighting}
\end{Shaded}

\begin{verbatim}
## Warning: Warning: fonts used in `flextable` are ignored because the `pdflatex`
## engine is used and not `xelatex` or `lualatex`. You can avoid this warning
## by using the `set_flextable_defaults(fonts_ignore=TRUE)` command or use a
## compatible engine by defining `latex_engine: xelatex` in the YAML header of the
## R Markdown document.
\end{verbatim}

\providecommand{\docline}[3]{\noalign{\global\setlength{\arrayrulewidth}{#1}}\arrayrulecolor[HTML]{#2}\cline{#3}}

\setlength{\tabcolsep}{2pt}

\renewcommand*{\arraystretch}{1.5}

\begin{longtable}[c]{|p{0.75in}|p{0.75in}}



\hhline{~~}

\multicolumn{1}{!{\color[HTML]{000000}\vrule width 0pt}>{\cellcolor[HTML]{CFCFCF}\raggedright}p{\dimexpr 0.75in+0\tabcolsep+0\arrayrulewidth}}{\fontsize{11}{11}\selectfont{\textcolor[HTML]{000000}{\textbf{Album Name}}}} & \multicolumn{1}{!{\color[HTML]{000000}\vrule width 0pt}>{\cellcolor[HTML]{CFCFCF}\raggedleft}p{\dimexpr 0.75in+0\tabcolsep+0\arrayrulewidth}!{\color[HTML]{000000}\vrule width 0pt}}{\fontsize{11}{11}\selectfont{\textcolor[HTML]{000000}{\textbf{Popularity}}}} \\



\endfirsthead

\hhline{~~}

\multicolumn{1}{!{\color[HTML]{000000}\vrule width 0pt}>{\cellcolor[HTML]{CFCFCF}\raggedright}p{\dimexpr 0.75in+0\tabcolsep+0\arrayrulewidth}}{\fontsize{11}{11}\selectfont{\textcolor[HTML]{000000}{\textbf{Album Name}}}} & \multicolumn{1}{!{\color[HTML]{000000}\vrule width 0pt}>{\cellcolor[HTML]{CFCFCF}\raggedleft}p{\dimexpr 0.75in+0\tabcolsep+0\arrayrulewidth}!{\color[HTML]{000000}\vrule width 0pt}}{\fontsize{11}{11}\selectfont{\textcolor[HTML]{000000}{\textbf{Popularity}}}} \\

\endhead



\multicolumn{1}{!{\color[HTML]{000000}\vrule width 0pt}>{\cellcolor[HTML]{EFEFEF}\raggedright}p{\dimexpr 0.75in+0\tabcolsep+0\arrayrulewidth}}{\fontsize{11}{11}\selectfont{\textcolor[HTML]{000000}{The Lion King (1994)}}} & \multicolumn{1}{!{\color[HTML]{000000}\vrule width 0pt}>{\cellcolor[HTML]{EFEFEF}\raggedleft}p{\dimexpr 0.75in+0\tabcolsep+0\arrayrulewidth}!{\color[HTML]{000000}\vrule width 0pt}}{\fontsize{11}{11}\selectfont{\textcolor[HTML]{000000}{76}}} \\





\multicolumn{1}{!{\color[HTML]{000000}\vrule width 0pt}>{\raggedright}p{\dimexpr 0.75in+0\tabcolsep+0\arrayrulewidth}}{\fontsize{11}{11}\selectfont{\textcolor[HTML]{000000}{Hercules (1997)}}} & \multicolumn{1}{!{\color[HTML]{000000}\vrule width 0pt}>{\raggedleft}p{\dimexpr 0.75in+0\tabcolsep+0\arrayrulewidth}!{\color[HTML]{000000}\vrule width 0pt}}{\fontsize{11}{11}\selectfont{\textcolor[HTML]{000000}{67}}} \\





\multicolumn{1}{!{\color[HTML]{000000}\vrule width 0pt}>{\cellcolor[HTML]{EFEFEF}\raggedright}p{\dimexpr 0.75in+0\tabcolsep+0\arrayrulewidth}}{\fontsize{11}{11}\selectfont{\textcolor[HTML]{000000}{The Little Mermaid (1997)}}} & \multicolumn{1}{!{\color[HTML]{000000}\vrule width 0pt}>{\cellcolor[HTML]{EFEFEF}\raggedleft}p{\dimexpr 0.75in+0\tabcolsep+0\arrayrulewidth}!{\color[HTML]{000000}\vrule width 0pt}}{\fontsize{11}{11}\selectfont{\textcolor[HTML]{000000}{65}}} \\





\multicolumn{1}{!{\color[HTML]{000000}\vrule width 0pt}>{\raggedright}p{\dimexpr 0.75in+0\tabcolsep+0\arrayrulewidth}}{\fontsize{11}{11}\selectfont{\textcolor[HTML]{000000}{Tarzan (1999)}}} & \multicolumn{1}{!{\color[HTML]{000000}\vrule width 0pt}>{\raggedleft}p{\dimexpr 0.75in+0\tabcolsep+0\arrayrulewidth}!{\color[HTML]{000000}\vrule width 0pt}}{\fontsize{11}{11}\selectfont{\textcolor[HTML]{000000}{1}}} \\



\end{longtable}

\end{document}
